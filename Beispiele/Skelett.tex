\begin{song}{title={Titel des Liedes}, lyrics={Der Dicher}, music={Der Musiker}}
	\begin{verse}
		D\chord{a}as hier\chord{e} ist d\chord{F}ie erste Str\chord{C}ophe,\\
		Man\chord{Esus4} schreibt \chord{H}die Akko\chord{G7}rde einfach\\
		ü\chord{E}ber de\chord{d}n Text.\chord{H} \\
	\end{verse}
	\begin{verse*}
		\chord{G}Das h\chord{F}ier is\chord{E}t auc\chord{D}h ein\chord{C}e Strophe,\\
		a\chord{G}lerdings o\chord{H}hne N\chord{c}ummer\\
	\end{verse*}
	\begin{refrain}
		Da\chord{Esus2}s ist der Refrain. Der funktioniert wie alle Strophen auch.\\
	\end{refrain}
	\begin{verse}
		Die Nummerierten strophen werden weitergezählt.\\
		Das hier ist also die zweite Strophe. Hier stehen\\
		keine Akkorde.\\
		N\chord{C}ur in d\chord{D}er le\chord{E}tzten zeile stehen Akkorde\\
	\end{verse}
	\begin{verse}
		D\chord{C#}ie Dritte S\chord{E5}trophe hat wieder Akk\chord{C}orde \chord{D}  \chord{Dsus2}      \chord{D}  \chord{e}  \\
		war\chord{C}um auch immer. vielleicht ist die Melodie \chord{a}anders.\\
	\end{verse}
	\begin{verse*}
		dieser Vers hat keine Akkorde. Beachte, dass durch den leerraum eine \\
		ne\chord{D}ue Strophe \chord{E}entsteht. Hier gibt es \chord{f#}akkorde.\\
		so schreibt man einen vers ohne Akkorde. \\
	\end{verse*}
\end{song}