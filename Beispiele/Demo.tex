\documentclass{article}

\usepackage{leadsheets} % für die Darstellung der Lieder
\usepackage[T1, OT1]{fontenc}
\usepackage{german}

%######### Konfiguration von leadsheets ##########
%Aussehen der Akkorde
\setchords{%
	format={\bfseries},%
	minor={lowercase},%Mollakkorde in kleinbuchstaben
	input-notation={german},%
	output-notation={german}}

% Definiere eine Umgebung refrain (statt chorus), um Ref.: vor dem refrain stehen zu haben.
\newversetype{refrain} [name={Ref.}, after-label={:}]

%### Template für den Titel: ###
% Alternativer Titel (Für lieder mit mehr als einem bekannten tiiel.
% kann zum beispiel als zusätzlicher Eintrag im Index erscheinen)
\definesongproperty{alt-title}

%## Template für den Titel (leadsheets liefert auch welche mit.)
\definesongtitletemplate{demo}{{\large\songproperty{title}} \hfill Worte: \songproperty{lyrics} Weise: \songproperty{music}}

%Aussehen der Seite
\setleadsheets{title-template=demo, %Template auswählen
	verse/numbered=true,%Automatische strophennummern. Stropfen ohne nummer mit verse*
	verses-label-format={\large\bfseries},% labels fett und groß schreiben
	info/format={\footnotesize\fontfamily{ptm}\selectfont\itshape}, %Zum erstellen von infoblocken (WIP)
	empty-chord-dim={1em},% Beite unter Akkorden ohne text
	align-chords={l}, %Akkorde linksbündig ausrichten
	smash-chords=false, % Zusätzlicher Platz um Worte, wenn der Akkord zu lang ist.
	before-song={}, % wird vor dem Titel ausgeführt
	after-title={},  % wird nach dem Titel ausgeführt
	after-song={},  % wird nach dem Text ausgeführt
	bar-shortcuts=true} %erlaubt die verwendung von | und : in liedtexten als Musikzeichen (siehe Kap 8.5 und 13)


\begin{document}
	\begin{song}{title={Titel des Liedes}, lyrics={Der Dicher}, music={Der Musiker}}
	\begin{verse}
		D\chord{a}as hier\chord{e} ist d\chord{F}ie erste Str\chord{C}ophe,\\
		Man\chord{Esus4} schreibt \chord{H}die Akko\chord{G7}rde einfach\\
		ü\chord{E}ber de\chord{d}n Text.\chord{H} \\
	\end{verse}
	\begin{verse*}
		\chord{G}Das h\chord{F}ier is\chord{E}t auc\chord{D}h ein\chord{C}e Strophe,\\
		a\chord{G}lerdings o\chord{H}hne N\chord{c}ummer\\
	\end{verse*}
	\begin{refrain}
		Da\chord{Esus2}s ist der Refrain. Der funktioniert wie alle Strophen auch.\\
	\end{refrain}
	\begin{verse}
		Die Nummerierten strophen werden weitergezählt.\\
		Das hier ist also die zweite Strophe. Hier stehen\\
		keine Akkorde.\\
		N\chord{C}ur in d\chord{D}er le\chord{E}tzten zeile stehen Akkorde\\
	\end{verse}
	\begin{verse}
		D\chord{C#}ie Dritte S\chord{E5}trophe hat wieder Akk\chord{C}orde \chord{D}  \chord{Dsus2}      \chord{D}  \chord{e}  \\
		war\chord{C}um auch immer. vielleicht ist die Melodie \chord{a}anders.\\
	\end{verse}
	\begin{verse*}
		dieser Vers hat keine Akkorde. Beachte, dass durch den leerraum eine \\
		ne\chord{D}ue Strophe \chord{E}entsteht. Hier gibt es \chord{f#}akkorde.\\
		so schreibt man einen vers ohne Akkorde. \\
	\end{verse*}
\end{song}
\end{document}